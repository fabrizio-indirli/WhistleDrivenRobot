The main goal that this project aims is to design a game where the Discovery microphone is used to control a robot, through a whistle. In particular, this board, after calculating the sound frequency, sends to the robot a related command for choosing the direction to take. The game consists in a race with obstacles where the winner will be the one who complete the path before the others.

\subsection{Problem statement}
The game consists in a race between two or more robots, where each player controls one of the machines through whistles using the related Discovery board. The most skilled player in the control of the robot will be the one who arrives first at the end of the path, gaining the victory. All impacts to the other player are allowed and warmly suggested.\\*
A specific movement of the robot will correspond to a precise frequency range:
\begin{itemize}
	\item Forward movement
	\item Backward movement
	\item Left movement
	\item Right movement
\end{itemize}

\subsection{Summary of the work}
The code is organized in three main sections, in order to improve the maintainability and the creation process:
\begin{itemize}
	\item \textit{The Transmitter}: this part is represented by the Discovery board and it converts the perceived sounds in numerical frequencies. After that, the board picks the expected command for that value and sends it to the robot.
	\item \textit{The Communication}: this is the part that transfers the command from the Discovery to the Arduino through a Bluetooth channel.
	\item \textit{The Receiver}: this is the real robot, which executes the received commands, changing its state.
\end{itemize}
