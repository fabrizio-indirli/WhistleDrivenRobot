The main goal of this project is to design a game where the user can remotely command a robot (the \textit{receiver}) by whistling in the microphone of a STM32 Discovery board (the \textit{transmitter}) . In particular, this board continuously acquires sound samples from its microphone and calculates their frequency, and then sends to the robot a related command for choosing the direction to take. The game consists in a race with obstacles where the winner will be the one who completes the path before the others.

\subsection{Problem statement}
The game consists in a race between two or more robots, where each player controls one of the machines through whistles using the related Discovery board. The most skilled player in the control of the robot will be the one who arrives first at the end of the path, gaining the victory. Collisions with other players are allowed and warmly suggested.\\*
A specific movement of the robot will correspond to a precise frequency range of the whistle:
\begin{itemize}
	\item Forward movement
	\item Backward movement
	\item Left movement
	\item Right movement
\end{itemize}

\subsection{Summary of the work}
The code is organized in three main sections, in order to improve the maintainability and the creation process:
\begin{itemize}
	\item \textit{The Transmitter}: this part is composed of the Discovery board and a HD44780 display; it converts the perceived sounds in numerical frequencies. Then, the board computes the command associated to that frequency, shows it on the display (along with the recognized frequency) and sends it to the robot through bluetooth.
	\item \textit{The Communication}: this is the part that transfers the command from the Discovery board to the Arduino board through a Bluetooth channel, using two paired HC-05 modules.
	\item \textit{The Receiver}: this is the real robot, which executes the received commands and changes its state accordingly.
\end{itemize}
