\subsection{Purpose}
The goal of this project is to design a game for two or more players, where the Discovery microphone is used to control a robot, through a whistle, detecting the frequency. In particular, this board, after calculating the frequency, sends to the robot a related command for choosing the direction to take, using a Bluetooth connection.

\subsection{Description of the project}
The robot has a cylindrical shape and it is equipped with wheels on the bottom that allow it to move forward, backward, turn left and right.\\*
The structure of the project is divided in two different part: the transmitter and the receiver.
The transmitter is represented by the STM32F4 discovery board which is in charge to recognize the frequency of the sounds emitted by the player, elaborate them into commands and send these commands to the receiver.
The receiver is represented by an Arduino board which is directly connected to the mechanical body of the robot. The Arduino board receives the commands from the discovery board and uses them to control the engines of the robot.
The wireless communication is managed through two hc-05 Bluetooth modules: one for the transmitter and one for the receiver.\\
The body of the robot was made with a 3D printer.

\subsection{Game logic}
The game consists of a race between two or more robots. Each player controls his own robot through whistles, and he will guide it along the way. The most skilled player in the control of the robot, the one who will arrive first at the end of the path, will be the winner.\\*
A specific movement of the robot will correspond to each sound (frequency):
\begin{itemize}
	\item Forward movement
	\item Backward movement
	\item Left movement
	\item Right movement
\end{itemize}

\subsection{Code structure}
The code is organized in three main sections, in order to improve the maintainability and the creation process:
\begin{itemize}
	\item \textit{The Transmitter}: this part is represented by the Discovery board and it converts the perceived sounds in numerical frequencies. After that, the board picks the expected command for that value and sends it to the robot.
	\item \textit{The Communication}: this is the part that transfers the command from the Discovery to the Arduino through a Bluetooth channel.
	\item \textit{The Receiver}: this is the real robot, which executes the received commands, changing its state.
\end{itemize}
